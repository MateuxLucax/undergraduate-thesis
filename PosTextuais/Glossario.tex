% ----------------------------------------------------------
% Glossário
% ----------------------------------------------------------

%Consulte o manual da classe abntex2 para orientações sobre o glossário.

%\glossary

% ----------------------------------------------------------
% Glossário (Formatado Manualmente)
% ----------------------------------------------------------

\chapter*{GLOSSÁRIO}
\addcontentsline{toc}{chapter}{GLOSSÁRIO}

{ \setlength{\parindent}{0pt} % ambiente sem indentação

\textbf{Árvore Sintática Abstrata}: É uma representação abstrata (simplificada) da estrutura semântica de um código fonte escrito em uma certa linguagem de programação.

\textbf{Ánalise Estática}: Tem por objetivo encontrar vulnerabilidades e demais problemas na aplicação, e normalmente é executada durante a fase de revisão de código dentro do ciclo de vida de desenvolvimento de sistemas. Idealmente, tais ferramentas encontrariam falhas de segurança automaticamente e com alto grau de confiança.

\textbf{Framework}: É um conjunto de bibliotecas, APIs e ferramentas que auxiliam o desenvolvedor a criar aplicações de forma mais rápida e eficiente.

\textbf{pub.dev}: É o repositório oficial de pacotes Dart e Flutter, onde desenvolvedores podem publicar e compartilhar pacotes com a comunidade.

\textbf{LSP}: Language Server Protocol, é um protocolo de comunicação entre editores de código e servidores de linguagem que permite a execução de análises estáticas e outras funcionalidades de forma mais eficiente.

\textbf{Plugin}: É um componente de software que adiciona uma funcionalidade específica a um programa maior.

\textbf{Lint}: Lint é o termo de ciência da computação para uma ferramenta de análise de código estático usada para sinalizar erros de programação, bugs, erros estilísticos e construções suspeitas.

\textbf{Linter}: É uma ferramenta de análise estática que entrega o "Lint" para uma linguagem de programação específica.

} % fim ambiente sem indentação

\chapter{Considerações Finais e Trabalhos Futuros}

Este trabalho teve como objetivo principal a criação de um pacote capaz de auxiliar desenvolvedores de aplicativos móveis utilizando o framework Flutter a criar aplicações mais acessíveis. Ademais, foi debatida a necessidade de difundir acessibilidade nas aplicações móveis, uma vez que a acessibilidade é um direito de todos e deve ser garantida em todas as aplicações. Soma-se também a falta de uma unificação nos requisitos de acessibilidade em ambas das principais plataformas móveis, Android e iOS, o que dificulta a criação de aplicações acessíveis, uma vez que é necessário um Mapeamento Sistemático da literatura para conseguir entender todos os requisitos de acessibilidade de ambas as plataformas e construir uma aplicação acessível.

A solução proposta baseia-se em um conjunto de regras compilado da documentação de ambas as plataformas \cite{iosaccessibility}, \cite{androidaccessibility}, que foi implementado em um \texttt{linter} utilizando a linguagem de programação Dart e o pacote \href{https://pub.dev/packages/custom_lint_builder}{custom\_lint\_builder}. O \texttt{linter} foi publicado no repositório do \href{https://pub.dev/packages/accessibility_lint}{pub.dev} e pode ser utilizado por qualquer desenvolvedor Flutter que busca criar aplicações mais acessíveis.

Ademais, o projeto foi estruturado de tal forma que permite evoluções e adição ou remoção de regras de acessibilidade conforme a necessidade. A documentação do pacote foi feita de forma a facilitar a utilização do mesmo por qualquer desenvolvedor Flutter. Todo o código-fonte do pacote está disponível no repositório do \href{https://github.com/MateuxLucax/accessibility-lint}{GitHub}. O mesmo possui uma licença Apache 2.0, o que permite a utilização e modificação do código-fonte do pacote por qualquer desenvolvedor. Com isso, futuros trabalhos podem ser feitos para aprimorar o pacote e adicionar novas funcionalidades.

Na perspectiva do autor, o pacote \texttt{accessibility\_lint} é uma ferramenta que pode ser utilizada por qualquer desenvolvedor Flutter que busca criar aplicações mais acessíveis. A acessibilidade é um direito de todos e deve ser garantida em todas as aplicações. A utilização do pacote pode auxiliar no desenvolvimento de aplicações mais acessíveis e garantir que a acessibilidade seja uma preocupação desde o início do desenvolvimento, podendo inclusive ser utilizada em ambientes de integração contínua.
\chapter{Fundamentação Teórica}

Neste capítulo, são descritos e detalhados os conceitos fundamentais e as ferramentas utilizadas durante a execução deste estudo. Inicialmente, procurou-se uma compreensão mais profunda das definições de acessibilidade, acessibilidade digital e acessibilidade universal, conceitos essenciais para a compreensão do escopo deste trabalho.

Em seguida, se realiza uma discussão acerca das vantagens advindas da incorporação da acessibilidade digital no processo de desenvolvimento de aplicações destinadas a dispositivos móveis.

Destaca-se a relevância deste aspecto e o impacto positivo que pode exercer sobre a experiência do usuário final. Visando também trazer uma compreensão das definições e impactos legais que a acessibilidade digital tem para os desenvolvedores, são discutidas as principais legislações vigentes no Brasil voltadas à defesa das PCD. 

Adicionalmente serão tratados quais as principais deficiências motoras, auditivas e visuais que afetam a população mundial, qual o impacto dela na vida dos portadores e quais os benefícios e relação da acessibilidade digital com a sua inclusão na sociedade.

Posteriormente, são apresentadas as tecnologias que se fizeram necessárias para a condução do desenvolvimento deste projeto. Nesta parte, dá-se ênfase à importância dos testes automatizados, procedimento que se revela fundamental para garantir a qualidade e efetividade das implementações realizadas.

Além disso, são discutidas as tecnologias específicas empregadas na elaboração de uma extensão para o Flutter que permita a análise estática de regras de acessibilidade em aplicações móveis desenvolvidas com este framework.

\section{Acessibilidade}

\subsection{Definição de Acessibilidade}

A acessibilidade é um princípio fundamental que visa garantir a todos os indivíduos, independentemente de suas limitações físicas, sensoriais, intelectuais ou psicossociais, a capacidade de acessar e utilizar adequadamente um produto, serviço ou ambiente. Neste estudo, o foco se volta à acessibilidade no âmbito digital, especialmente em relação aos dispositivos móveis.

Entretanto, é necessário que compreender que a acessibilidade é para todos, e que atos ou soluções simples, tem um grande impacto na qualidade de vida das pessoas. Assim como define \cite{kalbag2017} “acessibilidade no mundo físico, é o nível em que um ambiente é utilizável pelo maior número de pessoas possíveis”.

Sendo assim, é simples compreender a importância e benefícios da acessibilidade no mundo físico. \cite{kalbag2017} traz um excelente exemplo de uma mudança de acessibilidade que estamos tendo em nosso mundo e acabamos por nem perceber o seu real impacto. Conforme na figura 2, é possível visualizar que maçanetas pivotantes podem ser abertas facilmente com um movimento vertical para baixo, mesmo que tenham algum grau de deficiência motora, pois podem utilizar seu corpo ou outros objetos como apoio para realizarem a abertura da porta. A princípio, esse exemplo parece não ter muita conexão com o mundo digital, porém podemos utilizá-lo como uma base para refletir a forma que construímos nossa interface, onde adicionamos complicações visando a estética, ou então apenas por decisão de produto, mas que podem ter um impacto negativo na usabilidade de PCD.

\begin{figure}[!h]
	\centering
	\caption{"Maçanetas pivotantes (à esquerda) só requerem um leve empurrão por baixo. Maçanetas esféricas (à direita) necessitam de um movimento de torção com firmeza.”}
	\includegraphics[width=320pt]{Assets/Macanetas pivotantes.png}
	\fonte{\cite[p. 11]{kalbag2017}}
\end{figure}

\subsection{Legislação Brasileira}

A Lei Brasileira de Inclusão (LBI), também conhecida como Estatuto da Pessoa com Deficiência (Lei Nº 13.146, de 6 de Julho de 2015), é um marco legal que busca assegurar e promover, em condições de igualdade, o exercício dos direitos e das liberdades fundamentais por pessoa com deficiência, visando à sua inclusão social e cidadania.

A LBI aborda várias áreas importantes para a inclusão, incluindo a acessibilidade, que tem implicações diretas para a concepção e desenvolvimento de dispositivos móveis. Ela define acessibilidade como sendo a possibilidade e condição de alcance para utilização, com segurança e autonomia, de espaços, mobiliários, equipamentos urbanos, edificações, transportes, informação e comunicação, inclusive seus sistemas e tecnologias, bem como de outros serviços e instalações abertos ao público, de uso público ou privados de uso coletivo, tanto na zona urbana como na rural, por pessoa com deficiência ou com mobilidade reduzida.

Nesse contexto, a LBI define no Art. 3º inciso um que Acessibilidade é a “possibilidade e condição de alcance para utilização, com segurança e autonomia, de espaços, mobiliários, equipamentos urbanos, edificações, transportes, informação e comunicação, inclusive seus sistemas e tecnologias, bem como de outros serviços e instalações abertos ao público, de uso público ou privados de uso coletivo, tanto na zona urbana como na rural, por pessoa com deficiência ou com mobilidade reduzida”. \cite{lei13146}.

Adicionalmente, ela também define o que são tecnologias assistiva sou ajuda técnica no inciso três do Artigo 3 como sendo a “equipamentos, dispositivos, recursos, metodologias, estratégias, práticas e serviços que objetivem promover a funcionalidade, relacionada à atividade e à participação da pessoa com deficiência ou com mobilidade reduzida, visando à sua autonomia, independência, qualidade de vida e inclusão social” \cite{lei13146}.

O Capítulo II da LBI descreve quais os direitos de pessoas com deficiência. E no contexto das aplicações móveis temos no Art. 9º que a “disponibilização de recursos, tanto humanos quanto tecnológicos, que garantam atendimento em igualdade de condições com as demais pessoas.” \cite{lei13146}. Dessa forma é claro a necessidade de adequar as aplicações móveis para que não haja penalidades conforme descrito no Artigo 88 da LDI aos desenvolvedores no tangente a criação de barreiras para pessoas com deficiência.

Além da LBI, o Brasil conta com outras legislações voltadas à inclusão digital e à garantia dos direitos das pessoas com deficiência. Dentre elas, destacam-se o \cite{decreto5296}, que regulamenta as Leis nº 10.048 e 10.098 e estabelece normas e critérios para a promoção de acessibilidade das pessoas com deficiência, e a Lei Nº 12.965 \cite{lei12965}, mais conhecida como Marco Civil da Internet, que estabelece princípios, garantias, direitos e deveres para o uso da Internet no Brasil.

\subsection{Estado da Acessibilidade Digital}

A acessibilidade digital é um conceito que se refere à capacidade de pessoas com deficiência ou mobilidade reduzida de acessar e interagir com conteúdos, aplicativos e serviços digitais. Ela abrange uma ampla gama de tecnologias e práticas que visam tornar a experiência digital mais inclusiva e acessível a todos os usuários, independentemente de suas limitações.

Em uma pesquisa realizada por \cite{iaap}, os principais desafios na adição de acessibilidade no processo de desenvolvimento conforme a figura \ref{fig:desafios-acessibilidade-iaap} é a dificuldade na incorporação de acessibilidade nos estágios iniciais do desenvolvimento de software, tempo para implementar ou corrigir acessibilidade, treinamento e conhecimento das melhores prática e acesso a testadores que possuam deficiência. O curioso é que o orçamento é colocado como um dos menores desafios, o que pode ser um indicativo de que a acessibilidade é algo que empresas reconhecem como importante e estão dispostas a investir.

\begin{figure}[!ht]
	\centering
	\caption{Pensando especificamente na sua função no trabalho, quais desafios o seu programa de acessibilidade enfrenta?}
	\includegraphics[width=350pt]{Assets/IAAPDesafios.png}
	\label{fig:desafios-acessibilidade-iaap}
	\fonte{\cite[p. 16]{iaap}}
\end{figure}

\section{Flutter}

Flutter é uma estrutura de desenvolvimento de aplicativo que usa a linguagem de programação Dart. Ela permite a criação de interfaces de usuário altamente personalizáveis e expressivas com um bom desempenho. Ao contrário de outras estruturas de desenvolvimento móvel, como o React Native, Flutter não usa pontes de JavaScript, mas compila o código diretamente no código nativo do sistema operacional, o que melhora o desempenho \cite{flutter}.

Sendo assim, segundo a mantenedora o Flutter oferece melhor desempenho, capacidade de “Hot Reload” permitindo que os desenvolvedores experimentem e construam interfaces mais rapidamente, Personalização através de um rico conjunto de “Widgets” --- os “blocos de construção” do Flutter --- seguindo inicialmente diretrizes do Material Design para Android e Cupertino para iOS mas sem bloquear modificações do desenvolvedor e também, permite o desenvolvimento de aplicativos para diferentes plataformas partindo de uma única fonte de código.

Ademais, o Flutter oferece uma série de recursos que ajudam os desenvolvedores a criar aplicativos acessíveis. Ela suporta APIs de acessibilidade para Android e iOS, incluindo a API TalkBack para Android e a API VoiceOver para iOS. Além disso, o Flutter tem suporte para ampliação de tela, fontes maiores, contraste de cores suficiente e navegação por teclado \cite{flutter}. Entretanto ainda necessitam que o desenvolvedor utilize das ferramentas providas para criar aplicações realmente acessíveis.

\section{Análise Estática}

Análise Estática segundo \cite{interpreters} é uma técnica de verificação de programas que examina o código-fonte sem executá-lo, identificando possíveis erros, inconsistências e violações de regras de programação. Ela é amplamente utilizada em compiladores, ferramentas de análise de código e linters para garantir a correção e a qualidade do código.

Essa etapa no processamento de código fonte é usualmente utilizado na compilação ou interpretação de um código, entretanto, ferramentas de análise estática podem ser utilizadas para verificar a conformidade do código com padrões de codificação, identificar possíveis vulnerabilidades de segurança e otimizar a performance do código durante o processo de desenvolvimento.

Essa técnica será o pilar para a construção da extensão proposta neste trabalho, pois será responsável por analisar o código fonte de aplicações móveis desenvolvidas com Flutter e identificar possíveis violações de regras de acessibilidade e boas práticas de desenvolvimento em tempo real para auxiliar o desenvolvedor.

\subsection{Árvore Sintática Abstrata}

A construção de uma Árvore Sintática Abstrata (AST, do inglês Abstract Syntax Tree) é uma etapa fundamental no processo de compilação, sendo responsável por representar a estrutura de um programa de forma hierárquica e abstrata. De acordo com \cite{compiladores}, a AST organiza as construções da linguagem em nós que correspondem aos elementos essenciais do programa, eliminando detalhes sintáticos desnecessários, como pontuação e regras intermediárias presentes na gramática. Assim, a AST fornece uma visão condensada do programa, capturando sua lógica e estrutura de forma que facilite etapas subsequentes, como a análise semântica e a geração de código.

Com base nesses princípios, a AST e a análise estática são amplamente utilizadas não apenas em compiladores tradicionais, mas também em ferramentas modernas, como linters e otimizadores de código. Essas ferramentas aproveitam a representação abstrata fornecida pela AST para realizar diagnósticos e aplicar melhorias, garantindo que o código esteja em conformidade com padrões estabelecidos e otimizado para execução.

\pagebreak

\begin{figure}[!ht]
	\centering
  \caption[Código intermediário]{Exemplo de código intermediário \footnotemark}
	\includegraphics[width=350pt]{Assets/ExemploCompiladoresArvore.png}
	\label{fig:arvore-compiladores}
	\fonte{\cite[p. 26]{compiladores}}
\end{figure}

\footnote[1]{`\texttt{do i = i + 1; while ( a[i] < v; )}`.}

Como é possível ver na figura \ref{fig:arvore-compiladores} e descrito em \cite{compiladores} um laço de repetição é representado com um nó raiz do tipo "do-while", onde seu corpo e sua condição de parada são filhos. Os filhos do corpo é a operação de adição que está sendo realizada. Já os filhos da condição de parada representa a comparação entre o valor de um vetor e uma variável. Com isso é possível torar um código fonte e transformá-lo em uma representação mais abstrata e fácil de ser manipulada na forma de árvore.

\section{Mapeamento Sistemático da Literatura}

O Mapeamento Sistemático da Literatura (MSL) é uma técnica de pesquisa que visa identificar, avaliar e interpretar as evidências disponíveis em um determinado campo de estudo. Para \cite{srufrj}, o MSL é uma abordagem sistemática e rigorosa que permite mapear e analisar a literatura existente sobre um tema específico, identificando tendências, lacunas e oportunidades de pesquisa. Ao contrário de uma busca não ordenada e aleatória por informações em diferentes livros, testes, artigos e outros documentos.

Uma das principais vantagens da utilização do MSL é a possibilidade de filtrar e sintetizar um grande volume de informações, para \cite{researchsynthesis} o beneficio cresce em função do número de documentos de um tópico. Além disso, com as ferramentas que hoje existem como IEEE Xplore, Scopus, Web of Science, Google Scholar, entre outros, é possível realizar buscas mais precisas e abrangentes, garantindo que todas as fontes relevantes sejam consideradas.

Porém, conforme é destacado por \cite{srufrj} uma das principais limitações do MSL é a dependência da qualidade e da disponibilidade das fontes de informação. Se as fontes de informação não forem confiáveis ou não estiverem disponíveis, o MSL pode ser comprometido. Além disso, a seleção de critérios de inclusão e exclusão pode ser subjetiva e influenciar os resultados do MSL. Ademais, o realizador da pesquisa pode ser influenciado por um conhecimento maior sobre o tópico e acabar por não considerar fontes relevantes.

Diferentes metodologias e protocolos podem ser utilizados para conduzir um MSL, dependendo do objetivo da pesquisa e do campo de estudo. No entanto, a maioria dos MSLs segue uma estrutura semelhante, que envolve as seguintes etapas: definição da questão de pesquisa, seleção de fontes de informação, definição de critérios de inclusão e exclusão, extração de dados, análise dos resultados e apresentação dos achados.

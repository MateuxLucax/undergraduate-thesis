% ----------------------------------------------------------
% Pacotes básicos 
% ----------------------------------------------------------
\usepackage{amsmath}							% Pacote matemático
\usepackage{amssymb}							% Pacote matemático
\usepackage{amsfonts}							% Pacote matemático
%\usepackage{lmodern}							% Usa a fonte Latin Modern		
\usepackage{mathptmx} 							% Usa a fonte Times New Roman	 (UDESC/CCT)
\usepackage[T1]{fontenc}						% Selecao de codigos de fonte.
\usepackage[utf8]{inputenc}						% Codificacao do documento (conversão automática dos acentos)
\usepackage{lastpage}							% Usado pela Ficha catalográfica
\usepackage{indentfirst}						% Indenta o primeiro parágrafo de cada seção.
\usepackage[dvipsnames,table]{xcolor}			% Controle das cores
\usepackage{graphicx}							% Inclusão de gráficos
\usepackage{microtype} 							% para melhorias de justificação
\usepackage{lipsum}								% para geração de dummy text
\usepackage[brazilian,hyperpageref]{backref}	% Paginas com as citações na bibl
\usepackage[alf,abnt-emphasize=bf,abnt-full-initials=yes]{abntex2cite}					% Citações padrão ABNT
%\usepackage[num]{abntex2cite}					% Citações padrão ABNT numérica
\usepackage{adjustbox}							% Pacote de ajuste de boxes
\usepackage{subcaption}							% Inclusão de Subfiguras e sublegendas		
\usepackage{enumitem}							% Personalização de listas
\usepackage{siunitx}							% Grandezas e unidades
\usepackage[section]{placeins}					% Manter as figuras delimitadas na respectiva seção com a opção [section]
\usepackage{multirow}							% Multi colunas nas tabelas
\usepackage{array,tabularx} 					% Pacotes de tabelas
\usepackage{booktabs}							% Pacote de tabela profissonal
\usepackage{rotating}							% Rotacionar figuras e tabelas
\usepackage{xfrac}								% Fazer frações n/d em linha
\usepackage{bm}									% Negrito em modo matemático
\usepackage{xstring}							% Manipulação de strings
\usepackage{pgfplots}							% Pacote de Gráficos
\usepackage{tikz}								% Pacote de Figuras
\usepackage[american, cuteinductors,smartlabels, fulldiode, siunitx, americanvoltages, oldvoltagedirection, smartlabels]{circuitikz}						% Pacote de circuitos elétricos
\usepackage{chemformula}						% Pacote para fórmulas químicas
\usepackage{chngcntr}							% Pacte usado para deixar numeração de equações sequencial (UDESC/CCT)
\usepackage{amssymb} 							% Pacote matemático
\usepackage{pifont}							% Pacote de símbolos
\usepackage{makecell}							% Pacote de células de tabelas
\counterwithout{equation}{chapter}
% fonte: https://latex.org/forum/viewtopic.php?t=15392

% Comando para deixar numeração das equações contínua (1), (2), (3)... ao invés de organizar por capítulos (1.1)(1.2)... (2.1)(2.2)
%\renewcommand{\theequation}{\arabic{equation}}

%\numberwithin{equation}{section}


% Cabecalho cabeçalho somente com numeração de página 10pt
\makepagestyle{PagNumReduzida}
\makeevenhead{PagNumReduzida}{\ABNTEXfontereduzida\thepage}{}{}
\makeoddhead{PagNumReduzida}{}{}{\ABNTEXfontereduzida\thepage}
%fonte: https://github.com/abntex/abntex2/wiki/HowToCustomizarCabecalhoRodape
%fonte: Manual memoir seção 7.3 pg. 111 pdf http://linorg.usp.br/CTAN/macros/latex/contrib/memoir/memman.pdf 

% Personalização das opções das listas
\setlist[itemize]{leftmargin=\parindent}

% Citação online --- MODIFICAR ---
\newcommand{\citeshort}[1]{\citeauthoronline{#1}~(\citeyear{#1})}

\newcommand{\me}[1]{Elaborado pelo autor (#1).}

% Configuração do pgfplots
\pgfplotsset{compat=newest} %compat=1.14
\pgfplotsset{plot coordinates/math parser=false} 
\newlength\figureheight 
\newlength\figurewidth 

% Libraries do TiKz
\usetikzlibrary{quotes,angles,arrows}
\usetikzlibrary{through,calc,math}
\usetikzlibrary{graphs,backgrounds,fit}
\usetikzlibrary{shapes,positioning,patterns,shadows}
\usetikzlibrary{decorations.pathreplacing}
\usetikzlibrary{shapes.geometric}
\usetikzlibrary{arrows.meta}
\usetikzlibrary{external}

%\tikzexternalize[]
%\tikzexternalenable
%\tikzexternalize
%\tikzexternaldisable
%\tikzset{external/force remake}
%\tikzexternalize[shell escape=-enable-write18]

% Configurações do CircuiTiKz
\ctikzset{bipoles/thickness=1}
%\ctikzset{bipoles/length=1.2cm}
\ctikzset{monopoles/ground/width/.initial=.2}
\ctikzset{bipoles/resistor/height=0.25}
\ctikzset{bipoles/resistor/width=0.6}
\ctikzset{bipoles/capacitor/height=0.5}
\ctikzset{bipoles/capacitor/width=0.15}
\ctikzset{bipoles/generic/height=0.25}
\ctikzset{bipoles/generic/width=0.6}
%\ctikzset{bipoles/capacitor polar/length=0.5}
%\ctikzset{bipoles/diode/height=.375}
%\ctikzset{bipoles/diode/width=.3}
%\ctikzset{tripoles/thyristor/height=.8}
%\ctikzset{tripoles/thyristor/width=1}
\ctikzset{bipoles/vsourcesin/height=.5}
\ctikzset{bipoles/vsourcesin/width=.5}
\ctikzset{bipoles/cvsourceam/height=.6}
\ctikzset{bipoles/cvsourceam/width=.6}
%\ctikzset{tripoles/european controlled voltage source/width=.4}

\tikzstyle{every node}=[font=\footnotesize]
\tikzstyle{every path}=[line width=0.25pt,line cap=round,line join=round]
%\tikzstyle{every path}=[line cap=round,line join=round]


% Definição de cores MATLAB
\definecolor{matlab_blue}{rgb}	{         0,    0.4470,    0.7410}
\definecolor{matlab_orange}{rgb}{    0.8500,    0.3250,    0.0980}
\definecolor{matlab_yellow}{rgb}{    0.9290,    0.6940,    0.1250}
\definecolor{matlab_violet}{rgb}{    0.4940,    0.1840,    0.5560}
\definecolor{matlab_green}{rgb}	{	 0.4660,    0.6740,    0.1880}
\definecolor{matlab_lblue}{rgb}	{    0.3010,    0.7450,    0.9330}
\definecolor{matlab_red}{rgb}	{    0.6350,    0.0780,    0.1840}

% Personalização das legendas
\usepackage[format = plain, %hang
			justification = centering,
			labelsep = endash,
			singlelinecheck = false,
			skip = 6pt,
			listformat = simple]{caption}	

% Personalização das unidades
\sisetup{output-decimal-marker = {,}}
\sisetup{exponent-product = \cdot}
\sisetup{tight-spacing=true}
\sisetup{group-digits = false}

% Personalizações de tipo de colunas de tabelas
\newcolumntype{L}[1]{>{\raggedright\let\newline\\\arraybackslash\hspace{0pt}}m{#1}}
\newcolumntype{C}[1]{>{\centering\let\newline\\\arraybackslash\hspace{0pt}}m{#1}}
\newcolumntype{R}[1]{>{\raggedleft\let\newline\\\arraybackslash\hspace{0pt}}m{#1}}

% Personalizações de cores da UDESC
\definecolor{CapaAmareloUDESC}{RGB}{243,186,83}		% Especializacao
\definecolor{CapaVerdeUDESC}{RGB}{0,112,52}			% Mestrado
\definecolor{CapaVermelhoUDESC}{RGB}{171,35,21}		% Doutorado
\definecolor{CapaAzulUDESC}{RGB}{38,54,118} 		% Pós-Doutorado

% CONFIGURAÇÕES DE PACOTES
% Configurações do pacote backref
% Usado sem a opção hyperpageref de backref
\renewcommand{\backrefpagesname}{Citado na(s) página(s):~}
% Texto padrão antes do número das páginas
\renewcommand{\backref}{}
% Define os textos da citação
\renewcommand*{\backrefalt}[4]{
	\ifcase #1 %
	Nenhuma citação no texto.%
	\or
	Citado na página #2.%
	\else
	Citado #1 vezes nas páginas #2.%
	\fi}%

% alterando o aspecto da cor azul
%\definecolor{blue}{RGB}{41,5,195}

% informações do PDF
\makeatletter
\hypersetup{
	%pagebackref=true,
	pdftitle={\@title}, 
	pdfauthor={\@author},
	pdfsubject={\imprimirpreambulo},
	pdfcreator={LaTeX with abnTeX2},
	pdfkeywords={abnt}{latex}{abntex}{abntex2}{trabalho academico}, 
	colorlinks=true,       		% false: boxed links; true: colored links
	linkcolor=black,          	% color of internal links
	citecolor=black,        	% color of links to bibliography
	filecolor=black,      		% color of file links
	urlcolor=black,
	bookmarksdepth=4
}
\makeatother


\makeatletter
\newcommand{\includetikz}[1]{%
	\tikzsetnextfilename{#1}%
	\input{#1.tex}%
}
\makeatother


% ---
% Possibilita criação de Quadros e Lista de quadros.
% Ver https://github.com/abntex/abntex2/issues/176
%
\newcommand{\quadroname}{Quadro}
\newcommand{\listofquadrosname}{Lista de quadros}

\newfloat[chapter]{quadro}{loq}{\quadroname}
\newlistof{listofquadros}{loq}{\listofquadrosname}
\newlistentry{quadro}{loq}{0}

% configurações para atender às regras da ABNT
\setfloatadjustment{quadro}{\centering}
\counterwithout{quadro}{chapter}
\renewcommand{\cftquadroname}{\quadroname\space} 
\renewcommand*{\cftquadroaftersnum}{\hfill--\hfill}

\setfloatlocations{quadro}{hbtp} % Ver https://github.com/abntex/abntex2/issues/176
% ---


% Espaçamento depois do título
\setlength{\afterchapskip}{0.7\baselineskip}
% O tamanho do parágrafo é dado por:
\setlength{\parindent}{1.25cm}
% Controle do espaçamento entre um parágrafo e outro:
\setlength{\parskip}{0.0cm}  % tente também \onelineskip
%\SingleSpacing % Espaçamento simples 
\OnehalfSpacing % Espaçamento 1,5 (UDESC/CCT)
%\DoubleSpacing	% Espaçamento duplo

% ---
% Margens - NBR 14724/2011 - 5.1 Formato
% ---
\setlrmarginsandblock{3cm}{2cm}{*}
\setulmarginsandblock{3cm}{2cm}{*}
\checkandfixthelayout[fixed]
% ---


% To use externalize consider
%https://tex.stackexchange.com/questions/182783/tikzexternalize-not-compatible-with-miktex-2-9-abntex2-package
%Lauro Cesar digged into the problem until he came with a solution for me to test. And it Works!
%
%According to this link:
%
%The package calc changed the commands \setcounter and friends to be fragile. So you have to make them robust. The example below uses etoolbox with \robustify:
%
\usepackage{etoolbox}
\robustify\setcounter
\robustify\addtocounter
\robustify\setlength
\robustify\addtolength


%% How to silence memoir class warning against the use of caption package?
%% https://tex.stackexchange.com/questions/391993/how-to-silence-memoir-class-warning-against-the-use-of-caption-package
%\usepackage{silence}
%\WarningFilter*{memoir}{You are using the caption package with the memoir class}
%\WarningFilter*{Class memoir Warning}{You are using the caption package with the memoir class}

% --------------------------------------------------------
% INICIO DAS CUSTOMIZACOES PARA A UDESC
% --------------------------------------------------------

% --------------------------------------------------------
% Fontes padroes de part, chapter, section, subsection e subsubsection
% --------------------------------------------------------
% --- Chapter ---
\renewcommand{\ABNTEXchapterfont}{\fontseries{b}} %\bfseries
\renewcommand{\ABNTEXchapterfontsize}{\normalsize}
% --- Part ---
\renewcommand{\ABNTEXpartfont}{\ABNTEXchapterfont}
\renewcommand{\ABNTEXpartfontsize}{\LARGE}
% --- Section ---
\renewcommand{\ABNTEXsectionfont}{\normalfont}
\renewcommand{\ABNTEXsectionfontsize}{\normalsize}
% --- SubSection ---
\renewcommand{\ABNTEXsubsectionfont}{\fontseries{b}} %\bfseries
\renewcommand{\ABNTEXsubsectionfontsize}{\normalsize}
% --- SubSubSection ---
\renewcommand{\ABNTEXsubsubsectionfont}{\itshape}
\renewcommand{\ABNTEXsubsubsectionfontsize}{\normalsize}

\renewcommand{\ABNTEXsubsubsubsectionfont}{\normalfont}
\renewcommand{\ABNTEXsubsubsubsectionfontsize}{\normalsize}
% ---

% --------------------------------------------------------
% Fontes das entradas do sumario
% --------------------------------------------------------

\renewcommand{\cftpartfont}{\ABNTEXpartfont\selectfont}
\renewcommand{\cftpartpagefont}{\normalsize\selectfont}

\renewcommand{\cftchapterfont}{\ABNTEXchapterfont\selectfont}
\renewcommand{\cftchapterpagefont}{\normalsize\selectfont}

\renewcommand{\cftsectionfont}{\ABNTEXsectionfont\selectfont}
\renewcommand{\cftsectionpagefont}{\normalsize\selectfont}

\renewcommand{\cftsubsectionfont}{\ABNTEXsubsectionfont\selectfont}
\renewcommand{\cftsubsectionpagefont}{\normalsize\selectfont}

\renewcommand{\cftsubsubsectionfont}{\normalfont\itshape\selectfont}
\renewcommand{\cftsubsubsectionpagefont}{\normalsize\selectfont}

\renewcommand{\cftparagraphfont}{\normalfont\selectfont}
\renewcommand{\cftparagraphpagefont}{\normalsize\selectfont}

% --------------------------------------------------------
% Usando os pacotes hyperref, uppercase... 
% Para deixar a section do toc uppercase precisa de:
% --------------------------------------------------------
\usepackage{textcase}

\makeatletter

\let\oldcontentsline\contentsline
\def\contentsline#1#2{%
	\expandafter\ifx\csname l@#1\endcsname\l@section
	\expandafter\@firstoftwo
	\else
	\expandafter\@secondoftwo
	\fi
	{%
		\oldcontentsline{#1}{\MakeTextUppercase{#2}}%
	}{%
		\oldcontentsline{#1}{#2}%
	}%
}
\makeatother

% --------------------------------------------------------
% Renomenando as entradas de APÊNDICES E ANEXOS
% --------------------------------------------------------

\renewcommand{\apendicesname}{AP\^ENDICES}
\renewcommand{\anexosname}{ANEXOS}


% Manipulação de Strings
%\RequirePackage{xstring}

% Comando para inverter sobrenome e nome
\newcommand{\invertname}[1]{%
	\StrBehind{#1}{{}}, \StrBefore{#1}{{}}%
}%


% --------------------------------------------------------
% Alterando os estilos de Caption e Fonte
% --------------------------------------------------------
\makeatletter
% Define o comando \fonte que respeita as configurações de caption do memoir ou do caption
\renewcommand{\fonte}[2][\fontename]{%
	\M@gettitle{#2}%
	\memlegendinfo{#2}%
	\par
	\begingroup
	\@parboxrestore
	\if@minipage
	\@setminipage
	\fi
	\ABNTEXfontereduzida
	\configureseparator
	\captiondelim{\ABNTEXcaptionfontedelim}
	\@makecaption{#1}{\ignorespaces #2}\par
	\endgroup}


\captionstyle[\raggedright]{\raggedright}

\makeatother

\setlength{\cftbeforechapterskip}{0pt plus 0pt}
\renewcommand*{\insertchapterspace}{}

\newlength{\mylen}	% New length to use with spacing
\setlength{\mylen}{1pt}

\setlength{\cftbeforechapterskip}{\mylen}
\setlength{\cftbeforesectionskip}{\mylen}
\setlength{\cftbeforesubsectionskip}{\mylen}
\setlength{\cftbeforesubsubsectionskip}{\mylen}
\setlength{\cftbeforesubsubsubsectionskip}{\mylen}


% ---
% Ajuste das listas de abreviaturas e siglas ; e símbolos [Personalizada para UDESC com espaçamento 1,5]
% ---

% ---
% Redefinição da Lista de abreviaturas e siglas [Personalizada para UDESC com espaçamento 1,5]
\renewenvironment{siglas}{%
	\pretextualchapter{\listadesiglasname}
	\begin{symbols} 
		\setlength{\itemsep}{0pt}	% Ajuste para Espaçamento 1,5 (UDESC/CCT)
	}{% 
	\end{symbols}
	\cleardoublepage
}
% ---

% ---
% Redefinição da Lista de símbolos [Personalizada para UDESC com espaçamento 1,5]
\renewenvironment{simbolos}{%
	\pretextualchapter{\listadesimbolosname}
	\begin{symbols}
		\setlength{\itemsep}{0pt}	% Ajuste para Espaçamento 1,5 (UDESC/CCT)
	}{%
	\end{symbols}
	\cleardoublepage
}
% ---


% ---
% Remocao dos simbolos de < > das urls, ver manual pacote url pg 6 item 6
% https://github.com/abntex/biblatex-abnt/issues/16
\def\UrlLeft{}
\def\UrlRight{}
% ---

% ---
% FIM DAS CUSTOMIZACOES PARA A  Universidade do Estado de Santa Catarina - UDESC/CCT
% ---

\chapter{Desenvolvimento}

Esta seção tem como objetivo apresentar o processo de desenvolvimento da proposta atual, fundamentado nas etapas previamente descritas na seção de metodologia. Serão discutidos também os desafios encontrados durante o processo e possíveis ajustes na direção do projeto à medida que obstáculos emergem.

O processo de desenvolvimento da proposta inicia com a etapa de Mapeamento, na qual serão definidos os requisitos, formuladas as perguntas de pesquisa, realizado o mapeamento sistemático da literatura e estabelecidos os requisitos não funcionais para acessibilidade.

Em seguida, na etapa de Modelagem, serão construídos o Diagrama de Sequência, o Diagrama de Componentes, o Modelo Lógico de Dados e o Diagrama de Classes. Esses diagramas e modelos servirão como alicerce para a próxima fase.

Na fase do Projeto, será projetada a arquitetura de integração com o Flutter e selecionados os frameworks, bibliotecas e linguagens que serão usados no desenvolvimento da ferramenta.

Com a conclusão da etapa de Projeto, o processo de Desenvolvimento será iniciado, onde serão aplicados Padrões de Projeto e Expressões regulares, além de preparar a ferramenta para publicação.

Após a conclusão do desenvolvimento, a etapa de Testes será iniciada. Nela, será adotado o TDD, serão realizados Testes de Unidade e de Integração, bem como Testes dos Requisitos não funcionais de Acessibilidade.

Finalmente, na etapa de Validação, será realizada uma pesquisa com desenvolvedores e pessoas com deficiência, aplicado o SUS e executada uma prova de conceito. Essa etapa final permitirá avaliar a eficácia da ferramenta desenvolvida e identificar áreas de melhoria potencial. Cada uma dessas etapas será discutida em detalhes nas seções subsequentes deste capítulo.
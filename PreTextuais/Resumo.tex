% ---
% RESUMOS
% ---

% resumo em português
\setlength{\absparsep}{18pt} % ajusta o espaçamento dos parágrafos do resumo
\begin{resumo}

  Este trabalho tem como objetivo promover a acessibilidade entre desenvolvedores, conscientizando sobre a importância de tornar aplicações acessíveis, garantindo que todas as pessoas possam usufruir do que criamos. Baseando-se nas especificações de acessibilidade das principais plataformas móveis (Android e iOS), foi desenvolvido um pacote para aplicações Flutter, capaz de fornecer sugestões em tempo real sobre potenciais problemas de acessibilidade. O pacote foi implementado em Dart, a linguagem de programação do Flutter, e disponibilizado no repositório oficial de pacotes, o \href{https://pub.dev}{pub.dev}. Ele foi testado em aplicações de exemplo disponíveis em seu repositório oficial, demonstrando conformidade com os requisitos de acessibilidade estabelecidos. Dessa forma, desenvolvedores de aplicativos Flutter podem incorporar esta ferramenta em seus projetos, contribuindo para a criação de aplicações mais inclusivas.

  \textbf{Palavras-chave}: Acessibilidade. Dispositivos Móveis. Flutter. Análise Estática. Inspeção Contínua.
\end{resumo}
